 %%%%%%%%%%%%%%%%%%%%%%%%%%%%%%%%%%%%%%%%%
% "ModernCV" CV and Cover Letter
% LaTeX Template
% Version 1.3 (29/10/16)
%
% This template has been downloaded from:
% http://www.LaTeXTemplates.com
%
% Original author:
% Xavier Danaux (xdanaux@gmail.com) with modifications by:
% Vel (vel@latextemplates.com)
%
% License:
% CC BY-NC-SA 3.0 (http://creativecommons.org/licenses/by-nc-sa/3.0/)
%
% Important note:
% This template requires the moderncv.cls and .sty files to be in the same 
% directory as this .tex file. These files provide the resume style and themes 
% used for structuring the document.
%
%%%%%%%%%%%%%%%%%%%%%%%%%%%%%%%%%%%%%%%%%

%----------------------------------------------------------------------------------------
%	PACKAGES AND OTHER DOCUMENT CONFIGURATIONS
%----------------------------------------------------------------------------------------

\documentclass[11pt,a4paper,sans]{moderncv} % Font sizes: 10, 11, or 12; paper sizes: a4paper, letterpaper, a5paper, legalpaper, executivepaper or landscape; font families: sans or roman

\moderncvstyle{casual} % CV theme - options include: 'casual' (default), 'classic', 'oldstyle' and 'banking'
\moderncvcolor{black} % CV color - options include: 'blue' (default), 'orange', 'green', 'red', 'purple', 'grey' and 'black'

\usepackage{mathtools}
\usepackage{graphicx}
\usepackage{amsmath}
\usepackage{amsfonts}
\usepackage{booktabs}
\usepackage{tabularx}
\usepackage{xcolor}
\newcolumntype{Y}{>{\centering\arraybackslash}X}
\usepackage[scale=0.75]{geometry} % Reduce document margins
%\setlength{\hintscolumnwidth}{3cm} % Uncomment to change the width of the dates column
%\setlength{\makecvtitlenamewidth}{10cm} % For the 'classic' style, uncomment to adjust the width of the space allocated to your name

%----------------------------------------------------------------------------------------
%	NAME AND CONTACT INFORMATION SECTION
%----------------------------------------------------------------------------------------

\firstname{Antonio} % Your first name
\familyname{Cruciani} % Your last name

% All information in this block is optional, comment out any lines you don't need
\title{Curriculum Vitae}
%\address{Via Salti 78A}{Sant'Angelo in Pontano, Macerata, 62020}
%\mobile{3293094668 }
%\email{crc.antonio@gmail.com}
%\homepage{}{} % The first argument is the url for the clickable link, the second argument is the url displayed in the template - this allows special characters to be displayed such as the tilde in this example
%\extrainfo{additional information}
%\photo[70pt][0.4pt]{pictures/picture} % The first bracket is the picture height, the second is the thickness of the frame around the picture (0pt for no frame)
%\quote{"A witty and playful quotation" - John Smith}
\newcommand{\signature}[3][Antonio Cruciani]{%
 \vspace{4cm}
  \parbox{\textwidth}{
    \centering #3 \today\\
    \vspace{2cm}

    \parbox{7cm}{
      \centering
      \rule{6cm}{1pt}\\
       #1 
    }
    \hfill
 
  }
}
%----------------------------------------------------------------------------------------
%\photo[140pt][0.4pt]{img/photo3}
\begin{document}

%----------------------------------------------------------------------------------------
%	COVER LETTER
%----------------------------------------------------------------------------------------

% To remove the cover letter, comment out this entire block

%\clearpage

%\recipient{HR Department}{Corporation\\123 Pleasant Lane\\12345 City, State} % Letter recipient
%\date{\today} % Letter date
%\opening{Dear Sir or Madam,} % Opening greeting
%\closing{Sincerely yours,} % Closing phrase
%\enclosure[Attached]{curriculum vit\ae{}} % List of enclosed documents

%\makelettertitle % Print letter title

%\lipsum[1-2] % Dummy text
%\lipsum[4] % Dummy text

%\makeletterclosing % Print letter signature

%\newpage

%----------------------------------------------------------------------------------------
%	CURRICULUM VITAE
%----------------------------------------------------------------------------------------

\makecvtitle % Print the CV title

%----------------------------------------------------------------------------------------
%	EDUCATION SECTION
%----------------------------------------------------------------------------------------
\section{Contact Information}
\cventry{}{Email}{}{}{antonio.cruciani@gssi.it}{}
\cventry{}{Phone}{}{}{+39 3293094668}{}
\cventry{}{Address}{}{}{Viale Francesco Crispi, 7, L'Aquila (AQ), Italy}{}
\cventry{}{Web Site}{}{}{\href{https://antonio-cruciani.github.io/}{antonio-cruciani.github.io}}{}
\cventry{}{GitHub}{}{}{\href{https://github.com/Antonio-Cruciani}{github.com/Antonio-Cruciani}}{}
\cventry{}{LinkedIn}{}{}{\href{https://www.linkedin.com/in/antonio-cruciani-9b7b7083/}{linkedin.com/in/antonio-cruciani-9b7b7083}}{}
\cventry{}{dblp}{}{}{\href{https://dblp.org/pid/249/5159.html}{dblp.org/pid/249/5159}}{}
\section{Education}
\cventry{2020--Now}{Ph.D.}{GSSI - Gran Sasso Science Institute}{L'Aquila}{}{Ph.D., Computer Science\\
	\underline{Supervisors:} \textcolor{black}{\underline{{\href{http://www.mat.uniroma2.it/~pasquale/}{Prof. Francesco Pasquale}}}}, \textcolor{black}{\underline{{\href{https://www.gssi.it/institute/organization/item/10311-crescenzi-pierluigi}{Prof. Pierluigi Crescenzi}}}} }


\cventry{2017--2020}{Student}{University of Rome }{Tor Vergata}{\textit{Master's degree}}{Computer Science.\\
	\underline{Final mark} : 110/110 Cum Laude\\
	\underline{Supervisor}: \textcolor{black}{\underline{{\href{http://www.mat.uniroma2.it/~pasquale/}{Prof. Francesco Pasquale}}}}\\
	\underline{Thesis title}:Dynamic Random Graphs and unstructured P2P networks, analysis of two models inspired by the Bitcoin network.\\
	Available at the following \textcolor{black}{\underline{{\href{https://github.com/Antonio-Cruciani/dynamic-random-graph-generator}{link}}}}}


\cventry{2011--2017}{Student}{University of Rome }{Tor Vergata}{\textit{Bachelor's degree}}{Computer Science.\\ 
	\underline{Final mark} : 92/110\\
	\underline{Supervisor}:
	\textcolor{black}{\underline{ {\href{https://didattica.uniroma2.it/docenti/curriculum_vitae/4221-Giorgio-Gambosi}{Prof.Giorgio Gambosi.}}}}\\
	\underline{Thesis title}: Efficient learning methods for playlist prediction.\\
}


%\cventry{2006--2011}{High School }{ITIS Montani}{Fermo}{\textit{}}{ Qualified Industrial Technician specialization: Information and Technology}  % Arguments not required can be left empty




%\section{Masters Thesis}

%\cvitem{Title}{\emph{Money Is The Root Of All Evil -- Or Is It?}}
%\cvitem{Supervisors}{Professor James Smith \& Associate Professor Jane Smith}
%\cvitem{Description}{This thesis explored the idea that money has been the cause of untold anguish and suffering in the world. I found that it has, in fact, not.}

%----------------------------------------------------------------------------------------
%	WORK EXPERIENCE SECTION
%----------------------------------------------------------------------------------------
%\textcolor{blue}{\underline{{http://www.fub.it/index.php/it/Chi\%20siamo/Persone/Amati}{Giambattista Amati}}}
\newpage
\section{Experience}
\subsection{Research}
\cventry{August-October 2024 }{Visiting Ph.D. Student}{IIT Madras}{Working on distributed algorithms for highly dynamic graphs}  {}{\underline{Supervisor:}	\textcolor{black}{\underline{ {\href{http://www.cse.iitm.ac.in/~augustine/}{John Augustine}}}}}
\cventry{August 2023- March 2024 }{Visiting Ph.D. Student}{IIT Madras}{Working on distributed algorithms for highly dynamic graphs}  {}{\underline{Supervisor:}	\textcolor{black}{\underline{ {\href{http://www.cse.iitm.ac.in/~augustine/}{John Augustine}}}}}

\cventry{February- October\\ 2020 }{Big Data and Information Retrieval}{\textsc{ Big Data Analytics Lab at Fondazione Ugo Bordoni} }{Working on graph mining algorithms for distance functions estimation (\textcolor{black}{\underline{{\href{https://github.com/BigDataLaboratory/MHSE}{link}}}}), compression, clustering, centrality, and ranking algorithms}  {}{\underline{Supervisor:}	\textcolor{black}{\underline{ {\href{https://www.linkedin.com/in/gianni-amati-73974b6/}{Giambattista Amati}}}}}
\subsection{Teachings}

%\cventry{September 2019}{Workshop Presentation}{10th Italian Information Retrieval Workshop (IIR-2019)}{Padua}{}{I presented the joint project with Fondazione Ugo Bordoni about index compression techniques.}
\cventry{June 2019}{Seminar}{\textsc{University of Rome Tor Vergata}}{Talk on FPT Algorithms}{}{ I held a seminar about  Iterative Compression technique for NP-Hard problems on Graphs.}
\cventry{October 2018 June 2019}{Teaching Assistant}{\textsc{University of Rome Tor Vergata}}{ \href{http://www.didattica.uniroma2.it/docenti/curriculum/4044-Miriam-Di-Ianni}{Prof. Miriam Di Ianni}}{}{Computability and Computational Complexity Theory\\ {Link to the lessons material (IT) available at the following \textcolor{black}{\underline{ \href{https://github.com/Antonio-Cruciani/Esercitazioni-Informatica-Teorica}{link}}}}}

\cventry{December 2017 June 2018}{Teaching Assistant}{\textsc{University of Rome Tor Vergata}}{\href{https://didattica.uniroma2.it/docenti/curriculum/4952-Gianluca-Rossi}{Prof. Gianluca Rossi }}{}{Computer programming with laboratory}


\subsection{Work}


%------------------------------------------------
\cventry{October 2015 January 2016}{Developer}{\textsc{WeDot}}{Roma}{}{Software developer for Microsoft platforms, .Net , C\# ,Windows Server. 
}

\cventry{June-September 2010}{Intern}{\textsc{New System}}{Falerone,Fermo,Marche}{}{Web developer and sysadmin}




%------------------------------------------------

%\subsection{Miscellaneous}

%\cventry{2010--2011}{}{}{}{}{Spent some time finding myself. This was a courageous endeavour that didn't have a job title. It was quite important to my overall development though so I'm adding it to my CV. Also it explains the gap in my otherwise stellar CV.}

%\cventry{2009--2010}{Computer Repair Specialist}{Buy More}{Burbank}{}{Worked in the Nerd Herd and helped to solve computer problems. Allowed me to become expert in all forms of martial arts and weaponry.}
%----------------------------------------------------------------------------------------
%	AWARDS SECTION
%----------------------------------------------------------------------------------------
\section{Publications}
\subsection{Conferences}
\cvitem{2024}{A. Cruciani, MANTRA: Temporal Betweenness Centrality Approximation through Sampling. European Conference on Machine Learning and Principles and Practice of Knowledge Discovery in Databases (ECML-PKDD), Vilnius September 9-13. }
\cvitem{2023}{G. Amati, A. Cruciani, D. Pasquini, P. Vocca and S. Angelini, PROPAGATE: A Seed Propagation Framework to Compute Distance-Based Metrics on Very Large Graphs. European Conference on Machine Learning and Principles and Practice of Knowledge Discovery in Databases (ECML-PKDD), Turin September 18-22. }
\cvitem{2023}{R. Becker, P. Crescenzi, A. Cruciani and B. Kodric, Proxying Betweenness Centrality Rankings in Temporal Networks. 21st International Symposium on Experimental Algorithms (SEA), Barcelona July 24-26. }
\cvitem{2023}{A. Cruciani, F. Pasquale, Dynamic graph models inspired by the Bitcoin network-formation process. 24th nternational Conference on Distributed Computing and Networking (ICDCN), IIT Kharagpur January 4-7. }
\cvitem{2022}{A. Cruciani, F. Pasquale, Dynamic graph models for the Bitcoin P2P network: simulation analysis for expansion and flooding time. 24th International Symposium on Stabilization, Safety, and Security of Distributed Systems (SSS), Clermont-Ferrand November 15-17. (Brief Announcement)}
\subsection{Workshops}
\cvitem{2021}{P. Vocca, G. Amati, S. Angelini, A. Cruciani, G. Fusco, G. Gaudino and D. Pasquini, OASIS 2021, Topic modeling by community detection algorithms}
\cvitem{2019}{A. Cruciani, D. Pasquini, G. Amati, and P. Vocca, About Graph Index Compression Techniques, Proceedings of the 10th Italian Information Retrieval Workshop (IIR-2019), Padua, Italy, September 16-18, 2019, CEUR-WS.org/Vol-2441/paper23.pdf.}
%\newpage

\section{Schools}
\cvitem{March 2022}{Bertinoro International Spring School 2022 (\textcolor{black}{\underline{{\href{https://tempesta.cs.unibo.it/projects/BISS/2022/}{link}}}})}
\cvitem{September 2021}{European Summer School on Learning in Games, Markets, and Online Decision Making (\textcolor{black}{\underline{{\href{https://sites.google.com/a/diag.uniroma1.it/algadimar/european-summer-school-september-6-10-2021?authuser=0}{link}}}})}
\cvitem{July-August 2021}{Max Planck Advanced Course on the Foundations of Computer Science (Convex Optimization)(\textcolor{black}{\underline{{\href{https://conferences.mpi-inf.mpg.de/adfocs-22/}{link}}}})}
\cvitem{May - June 2021}{Algorithmic Tools for Massive Network Analytics (\textcolor{black}{\underline{{\href{https://sites.google.com/view/algtools}{link}}}})}
\cvitem{August 2020}{Max Planck Advanced Course on the Foundations of Computer Science (Market Design and Computational Fair Division)(\textcolor{black}{\underline{{\href{https://conferences.mpi-inf.mpg.de/adfocs/}{link}}}})}
\section{Advanced Courses}
\cvitem{2019}{Semidefinite Programming and Discrete Optimization. University of Rome: \lq\lq Tor Vergata\rq\rq. Ph.D. (Computer Science, Control and
	Geoinformation) course held by \textcolor{black}{\underline{ \href{http://wwwu.aau.at/anwiegel/}{Prof. Angelika Wiegele}}}.}
\cvitem{2019}{Natural Distributed Algorithms. University of Rome: \lq\lq Tor Vergata\rq\rq. Course held by \textcolor{black}{\underline{ \href{https://www-sop.inria.fr/members/Emanuele.Natale/}{Dr. Emanuele Natale}}}.}
\cvitem{2019}{Algorithms and computational models for large-scale data analysis. University of Rome: \lq\lq La Sapienza\rq\rq. Ph.D. (Data Science) course held by \textcolor{black}{\underline{ \href{https://sites.google.com/site/silviolattanzi/}{Silvio Lattanzi}}}.}


\section{Certifications}
\cvitem{2017}{[MOOC] Approximation Algorithms by \'Ecole Normale Sup\'erieure }
\cvitem{}{Massive open online course by ENS on approximation algorithm. Particularly emphasizes algorithms that can be designed using linear programming and semidefinite programming.}
\cvitem{2017}{\href{https://www.coursera.org/account/accomplishments/specialization/certificate/8ENFUNVFRNSZ}{Machine Learning Specialization by Washington University}}
\cvitem{}{Online specialization on machine learning covering: foundations of ML, regression, classification, clustering and retrieval. To see the certification click on the name of specialization }
\cvitem{2017}{\href{https://www.coursera.org/account/accomplishments/certificate/8E6BSNQNMR56}{Machine Learning By Stanford University}}
\cvitem{}{Online course on machine learning, topics: supervised learning, unsupervised learning. To see the certification click on the name of specialization }
%\cvitem{2017}{\href{https://drive.google.com/file/d/0Bwk56PFyrBW_TWFUOG9ocS1fZ3M/view}{Common European Framework (CEFR) B1}}
%\sqrt{}


%----------------------------------------------------------------------------------------
%	COMPUTER SKILLS SECTION
%----------------------------------------------------------------------------------------
%
\section{Programming skills}

\cvitem{Basic}{\textsc{owl}, \textsc{sparql},\textsc{fortran},\textsc{cobol},\textsc{lisp}}
\cvitem{Intermediate}{\textsc{go},\textsc{matlab},\textsc{javascript},\textsc{r},\textsc{asp.net},\textsc{java}}
\cvitem{Advanced}{\textsc{python},\textsc{julia},\textsc{java},\textsc{c},\textsc{c++},\textsc{c\#},\textsc{sql},\textsc{php}}
\cvitem{Frameworks}{Apache Spark}

%----------------------------------------------------------------------------------------
%	COMMUNICATION SKILLS SECTION
%----------------------------------------------------------------------------------------

%\section{Communication Skills}

%\cvitem{2010}{Oral Presentation at the California Business Conference}
%\cvitem{2009}{Poster at the Annual Business Conference in Oregon}

%\newpage
%----------------------------------------------------------------------------------------
%	LANGUAGES SECTION
%----------------------------------------------------------------------------------------

\section{Languages}

\cvitemwithcomment{Italian}{Mother tongue}{}
\cvitemwithcomment{English}{Fluent}{}

%\cvitemwithcomment{Dutch}{Basic}{Basic words and phrases only}

%----------------------------------------------------------------------------------------
%	INTERESTS SECTION
%----------------------------------------------------------------------------------------

\section{Interests}

\renewcommand{\listitemsymbol}{-~} % Changes the symbol used for lists
\cvlistdoubleitem{Graph Mining}{Distributed Computing}
\cvlistdoubleitem{Temporal Graphs}{Randomized Algorithms}
\cvlistdoubleitem{Random Graphs}{Approximation Algorithms}
\cvlistdoubleitem{Evolving Graphs}{Statistical Learning}

%\signature{}{Antonio Cruciani}
%\section{Autorizzazione trattamento dati personali}
%Autorizzo il trattamento dei miei dati personali ai sensi del Decreto Legislativo 30 giugno 2003, n. 196 -Codice in materia di protezione dei dati personali-
\end{document}